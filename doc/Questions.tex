\documentclass[11pt]{beamer}
\usepackage[utf8]{inputenc}
\usepackage[T1]{fontenc}
\usepackage{lmodern}
\usepackage{amsmath}
\usepackage{amsfonts}
\usepackage{amssymb}
%\usepackage{enumitem}
\usepackage{graphicx}
\usetheme{EastLansing}
\author{Wang Yao}
\begin{document}
	\title{Data Science Study Notes}
	\subtitle{Some Common Interview Questions}
	%\logo{}
	%\institute{}
	%\date{}
	%\subject{}
	%\setbeamercovered{transparent}
	%\setbeamertemplate{navigation symbols}{}
	\begin{frame}[plain]
	\maketitle
\end{frame}

\begin{frame}
\frametitle{Correlation and PCA in dimension reduction}
\begin{block}{Question:}
	What are basic methods to reduce dimension?
\end{block}
\begin{itemize}
	\item Use correlation test to remove correlated numerical variables with business understanding
	\begin{itemize}
		\item A correlation coefficient measures the extent to which two variables tend to change together. The coefficient describes both  strength and direction of the relationship
		\item \textbf{Pearson correlation} evaluates the \textbf{linear relationship} between two \textbf{continuous variables}
		\item \textbf{Spearman rank-order correlation} evaluates the \textbf{monotonic relationship} between two continuous or \textbf{ordinal} variables. In a \textbf{monotonic relationship}, the variables tend to change together, but not necessarily at a constant rate. 
		\item The \textbf{Pearson} and \textbf{Spearman} coefficient are both nearly $0$ \textbf{does not} implies there is no strong relationship between two variables
	\end{itemize}
\end{itemize}
\end{frame}

\begin{frame}
\frametitle{Correlation and PCA in dimension reduction}
\begin{itemize}
	\item  Use Chi-Square test to remove correlated categorical variables with business understanding
\end{itemize}
\end{frame}

\begin{frame}
\frametitle{Correlation and PCA in dimension reduction}
\begin{itemize}
	\item Use PCA and pick the components which can explain the maximum variance in the data set. 
\end{itemize}
\end{frame}

\end{document}