\documentclass[10.5pt]{res} % default is 10 pt
%\usepackage{helvetica} % uses helvetica postscript font (download helvetica.sty)
%\usepackage{newcent}   % uses new century schoolbook postscript font
\setlength{\textheight}{9.5 in} % increase text height to fit resume on 1 page
\newsectionwidth{16pt}  % So the text is indented by 16pt under section headings

\usepackage{marvosym}
%\usepackage{amssymb}
%\usepackage{amsfonts}
%\usepackage{hyperref}
%\usepackage{savetrees}
%\usepackage[compact]{titlesec}

\begin{document}

\name{Wang Yao\\[5pt]} % the \\[5pt] adds a blank line after name
\address{%{LinkedIn:~~\url{http://www.linkedin.com/pub/wang-yao/21/586/2b1}\\
\quad Phone: (732) 447-6825~~E-mail:~~yaowang74\footnotesize{\MVAt}\normalsize{gmail.com}
~~Work Authorization: U.S. Citizen}%\\
%\qquad\qquad\qquad\qquad\quad Address: 707 Margaret Ct, South Plainfield, NJ, 07080}


\begin{resume}

%\section{OBJECTIVE}
%    To obtain a full-time position that utilizes a strong background in operations research and computer programming%in the pharmaceutical industry that utilizes a strong background in applied mathematics and computer programming %operations research or computational mathematics

\section{SUMMARY}
5+ years experiences in data science and machine learning, a well-trained scientist in mathematical optimization and algorithm development, passionate presenter with extensive knowledge in numerical methods for solving large scale statistical learning problems. Highly analytical, focused and personable, with broad interests in quantitative research, technology and consulting

%\section{SPECIALTIES \& INTERESTS}
%    Mathematical Modeling; Machine learning; Big data optimization; Statistical analysis; Computational mathematics%Parallel computing; Convex analysis
\section{EDUCATION}
  %\textbf{Rutgers University, New Brunswick, NJ.} \\
                % \sl will be bold italic in New Century Schoolbook (or any postscript font) and just slanted in Computer Modern (default) font
                %\textbf{Rutgers University}, New Brunswick, New Jersey\\
                \textbf{Ph.D.} in \textbf{Operations Research}, \textbf{Rutgers University, New Brunswick, NJ}\hfill 2010-2016\\
                \textbf{B.A.} in \textbf{Mathematics}, Minor: \textbf{Statistics}, \textbf{Rutgers University, NJ}. \emph{Magna Cum Laude}\hfill 2006-2010
                %Minor: \textbf{Statistics} % \emph{Magna Cum Laude}
                %\textbf{Rutgers University, New Brunswick, New Jersey}
                %Rutgers University, New Brunswick, NJ.
                %Minor: \textbf{Statistics}.
%
\section{RESEARCH} \parskip 11pt
	\textbf{Ph.D. Dissertation}: Approximate Versions of Alternating Direction Method of Multipliers
	\begin{itemize}\itemsep -1pt
		\item Developed three new %, more practically verifiable 
		 numerical algorithms for multi-block minimization models in machine learning  %, so that each subproblem can be solved approximately% and hence the elapsed time is significantly decreased
		\item Implemented new engine for common optimization problems with Matlab and C++, and wrote a library including conjugate gradient, L-BFGS and steepest decent method as sub-solvers
		\item Overall computing overhead is reduced by 30\% compared with classical frameworks
	\end{itemize}
%
\section{WORK EXPERIENCE} \parskip 11pt
	\textbf{Honeywell} \quad Data Scientist, Center of Excellence\hfill Oct. 2016 - Present
		\begin{itemize}
			\item 
		\end{itemize}
	\textbf{Chubb Corporation}\qquad Claim Actuarial and Advanced Analytics\hfill May 2015 - Oct. 2016
		\begin{itemize}
			%\item Learned \textbf{JSON} and \textbf{Solr} and enhanced the text mining web application. Proficient in \textbf{Git} 
			\item Performed ETL on massive claims data to create variables. Built decision tree severity model for 7-Eleven bodily injury claims at first contact and logistic regression model at 6 months. Compared to subjective classification that is used in practice, new models leads to significant improvements in the prediction of severity level and insurer's reserves 
			%\item Many companies estimate (and therefore reserve) bodily injury compensation directly from initial medical reports. This practice may underestimate the final cost, because the severity is often assessed during the recovery period. Since the evaluation of this severity is often only qualitative, in this paper we apply an ordered multiple choice model at different moments in the life of a claim reported to an insurance company. We assume that the information available to the insurer does not flow continuously, because it is obtained at different stages. Using a real data set, we show that the application of sequential ordered logit models leads to a significant improvement in the prediction of the BI severity level, compared to the subjective classification that is used in practice. We also show that these results could improve the insurer’s reserves notably.
			%\item Extracted, Transformed and loaded large amount of the messy data from multiple sources with \textbf{SAS} Enterprise Guide
			\item  Carried out the statistical analysis 
					 on social network data of underwriters with \textbf{R} and \textbf{iGraph} to identify the key network metrics shared by successful underwriters. Established the SVM network-revenue model and provided actionable solutions to help underwriters discover more revenue opportunities by improving their professional network structure
					%by calculating various networking metrics for each underwriter and compare their professional network structure
					%\item Established the preliminary network-revenue model and provided actionable solutions to help underwriters identify more revenue opportunities by improving their professional network structure 
					%provided actionable solution for underwriters to create more revenue by using and expanding their professional network 
		\end{itemize}
	\textbf{Novartis Pharmaceuticals Corporation}\quad Intern, Integrated Quantitative Science   \hfill Summer 2014
		\begin{itemize}\itemsep -1pt
			%\item Implemented a web application 
%			\item Programmed the effective plotting interface for understanding and interpreting modeling work
			\item Learned basic PK/PD models, applied quantitative methods  for dose escalation and selection in clinical  trials. Developed and designed web application for clinicians to compare simulation results and test hypothesis using \textbf{Linux}, \textbf{Apache} web server, along with \textbf{MySQL}, \textbf{Python} and \textbf{R} on back-end;  \textbf{Javascript}  and \textbf{D3.js} for front-end visualization. Greatly reduced the overhead of communication between clinicians and  pharmacometricians					
			%\item Developed and designed web application for clinicians to compare simulation results and test hypothesis using \textbf{Linux}, \textbf{Apache} web server, along with \textbf{MySQL}, \textbf{Python} and \textbf{R} on back-end;  \textbf{Javascript}  and \textbf{D3.js} for front-end visualization   
			%\item Greatly reduced the overhead of communication between clinicians and pharmacometricians
		\end{itemize}
    \textbf{Eli Lilly and Company} Data Visualization Intern, Manufacturing Technology\hfill Summer 2012, 2013
        \begin{itemize}\itemsep -1pt
           	\item Tested extensively the Bioprocess Data Collection System Data Mart(BDCSDM) with \textbf{SQL} and enhanced the data visualization application by adding new practical features with \textbf{.NET}
            \item Significantly advanced automation level of visualization tool  from previous internship to minimize the effort of maintenance. Independently developed an automatic data acquisition program for five filtration experiments
%           \item Significantly advanced the level of automation and user-friendliness of Excel data visualization application to minimize the effort of maintenance %the data visualization tool
%           \item Investigated and resolved a data communication issue, securing correct data output%Probe and successfully fix the indication error in the new data acquisition system that may cause serious consequences
%           \item Leveraged LabView software to independently develop and test an application that automatically allows execution and data acquisition of five simultaneous filtration experiments
			%\item Independently developed an automatic data acquisition program for five filtration experiments
%           \item Fully utilized the advanced equipment by providing programming support and hence greatly reduced the cost of human resources
            \item Fully utilized the advanced equipment and hence greatly reduced the cost of human resources and created manuals for both data visualization and acquisition applications to aid users and developers
            %\item Created manuals for both data visualization and acquisition applications to aid users and developers
        \end{itemize}
    \textbf{Command, Control, and Interoperability Center for Advanced Data Analysis}  \hfill Jan.-Jun. 2013
        \begin{itemize}\itemsep -1pt
            \item Captured and analyzed complex information given by Coast Guard to properly frame the problem and helped to establish the mathematical model with \textbf{Xpress-Mosel}%that generates various schemes with different parameters
            %\item Presented the briefings to upper management monthly and drafted the guidance on the project
            \item Presented the briefings to upper management monthly and drafted the guidance on the project. Provided best sharing plan for U.S. Coast Guard to cover required mission hours under tight budget
        \end{itemize}
%   \textbf{Eli Lilly and Company} Data Visualization Intern, Manufacturing Technology \hfill Summer 2012
%       \begin{itemize}\itemsep -1pt
%           \item Tested extensively the Bioprocess Data Collection System Data Mart(BDCSDM) with \textbf{SQL} and enhanced the data visualization application by adding new practical features with \textbf{.NET}
%           %\item Enhanced the Excel data visualization application by adding new practical features with \textbf{.NET}
%           \item Upgraded the whole interface to be database (BDCSDM) driven, more user-friendly and maintenance free
            %\item Exploited Excel's programming graphical capability to effectively deliver more information % and comprehensively
%       \end{itemize}
%   \textbf{American International Group, Inc.} Project Consultant \hfill March 2012
%       \begin{itemize}\itemsep -1pt
%           \item Conducted statistical analysis with \textbf{R} on data of health insurance claims of college students
%           \item Identified the patterns and described the trends in student health market for different states
%           \item Worked closely with PPO director and supported underwriter to adjust pricing policy
%       \end{itemize}
%   Project Title: \textbf{New Algorithms for Solving LASSO Regression}    \hfill Fall 2011-Spring 2012%\\
%       \begin{itemize}\itemsep -1pt
%           %\item Discovered a new application in biostatistics for alternating direction method of multipliers (ADMM)
%           \item Applied new robust algorithm for LASSO that well suited to big data and parallel computing
%           \item Put the advanced termination criteria into practice to significantly shorten the running time
%           \item The result was presented at annual INFORMS conference at Charlotte, North Carolina
%       \end{itemize}
%   Advisor: \textbf{Professor Jonathan Eckstein}, Rutgers Center for Operations Research\\
%      Project Title: \textbf{Logical Analysis of Data: Maximum Positive Patterns in Dataset}\hfill Spring 2011
%      Mentor:\textbf{Professor Endre Boros}, Rutgers Center for Operations Research
%          \begin{itemize}  \itemsep -1pt % reduce space between items
               %\item  Studied the logic-based methodology for analyzing data, which allows us to detect the hidden patterns in the observations                 by exploring the minimal subset of data
%               \item  Established the nonlinear integer programming model for finding the pattern of a large data set% and eliminating redundant constraints.
%               \item Simplified the model through linearization and solved with MOSEL, the result is highly competitive% with CLIP3, which is a widely used machine learning algorithm
%               \item  Discussed the progress with mentor regularly and presented the project at RUTCOR
%          \end{itemize}
%   Project Title: \textbf{Bounding the Probabilities and Its Application to Finance} \hfill Fall 2010    %\\
%   Mentor:\textbf{Professor Andr\'{a}s Pr\'{e}kopa}, Rutgers Center for Operations Research
%      \begin{itemize}  \itemsep -1pt % reduce space between items
%          \item Established the model to describe a stock's price by using geometric Brownian motion.%Learned to use the geometric %                Brownian motion as model to describe a stock's price
%          \item Utilized Brownian motion to model a stock's behavior and computed the probability of its price falling into the given lower and upper bound during a time interval %compared the results with other from discrete optimization and linear programming.
%          \item Derived the second-order bound and Hunter's upper bound for the objective probability with knowledge of discrete %optimization, and linear programming
%          \item Successfully derived the formulas and implemented the algorithms in Matlab, applied them to the real data and presented the results at Rutgers Center for Operations Research
%      \end{itemize}
%  {\textbf{Research Experience for Undergraduates, DIMACS}}\hfill Summer 2008 \\
%               Project Title: \textbf{Extreme Value Theory: Application to Aviation Safety}\hfill Summer 2008
%               Mentor:\textbf{Professor Regina Liu}, Department of Statistics, Rutgers University
%                \begin{itemize}  \itemsep -1pt % reduce space between items
%               \item  Generated the random samples to represent the landing data of aircrafts of different distributions, computed the estimators and the estimates of the tail quantile with different critical levels. %plot the histograms and draw the empirical distribution functions
%               \item  Used ``\textbf{R}'' to generate the random samples (to represent the landing data of aircrafts) of different distributions, computed the estimators and the estimates of the tail quantile corresponding to different critical levels. %plot the histograms and draw the empirical distribution functions
%               \item  Computed the estimators and the estimates of the tail quantile of those distributions corresponding to different critical levels. % $\alpha = 0.05,0.01, 0.001$(in regulation of FAA, $\alpha=10^{-7}$).
%               \item  Determined, based on generated aircraft landing data, a suitable threshold point which can guarantee that the probability of an aircraft lands beyond the threshold point is less than $0.001$ %The goal of the project is to determine, based on given aircraft landing data, a suitable threshold point which can guarantee that the probability of an aircraft lands beyond the threshold point is less than $10^{-7}$.  Extreme Value Theory is ideally suited for tackling this problem.
%               \item  Gave two formal presentations on the project to the whole group of Center for Discrete Mathematics and Theoretical Computer Science(DIMACS) %Learned the Extreme Value Theory and explored some applications in science and engineering.
%               \end{itemize}
%	\section{SKILLS} \begin{itemize} \itemsep -1pt
%                 \item \textbf{R}, \textbf{Python}, \textbf{SAS}, \textbf{Java}, \textbf{C}++,  \textbf{SQL},  \textbf{VB.NET},
%                 \textbf{Javascript}, \textbf{Eclipse}, \textbf{Visual Studio}, \textbf{Linux}
                 %\item   %\textbf{NONMEM}, \textbf{Spotfire}, \textbf{IPython}, \textbf{RShiny}, \textbf{SPSS}, \textbf{Minitab}  \textbf{AMPL}, \textbf{MOSEL}, \textbf{PHP}, \textbf{Matlab}, %and %
%                 \item \textbf{Strong communication}, \textbf{analytical} and \textbf{presentation} skills. Native in Chinese%Native speaker of Chinese.Fluent in both \textbf{spoken and written English}.
%                 \end{itemize}
%	\section{OTHER WORK EXPERIENCE}
%	Job Title: \textbf{Teaching Assistance}, Department of Mathematics, Rutgers University\hfill 2011-2012
%		\begin{itemize} \itemsep -1pt
%			\item Assisted students to learn the materials by conducting recitations and providing office hours weekly
%			\item Inspired their learning enthusiasm through encouraging students to ask and solve questions
%			\item Communicated with supervising faculty periodically to discuss any problem of the class
%	\end{itemize}
%	\section{HONORS \& AWARDS}  \textbf{Excellence Fellowship} \qquad \qquad Rutgers Center for Operations Research  \hfill 2009-2011
%           \newline \textbf{Dean's List} \qquad \qquad \qquad \qquad \qquad \qquad \qquad Rutgers College \hfill 2007-2009
%           \newline\textbf{Delta Epsilon Iota}\qquad \qquad ~~Academic Honor Society at Rutgers University \hfill 2007-Present
%           \newline\textbf{Pi Theta Kappa}\qquad \qquad \qquad \qquad \quad ~~ ~~ National Honor Society  \hfill  2006-Present
\end{resume}
\end{document} 